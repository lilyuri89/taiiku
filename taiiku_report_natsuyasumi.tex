\documentclass[10.5pt]{jarticle}

\usepackage[top=35mm, bottom=30mm, left=30mm, right=30mm, includefoot]{geometry}

\begin{document}

\section{歴史}
\subsection{日本への伝来}
1871年に来日した米国人ホーレス・ウィルソンが当時の東京開成学校予科(現在の東京大学)で教え、その後「打球おにごっこ」という名で全国的に広まった。従って、日本国内の野球の創世記の歴史は、そのまま大学野球の創世記の歴史と重なっている(詳細については当該記事を参照のこと)。なお、ホーレス・ウィルソンは2003年、その功績から日本野球殿堂入り(新世紀表彰)している。

1878年、平岡ひろしが日本初の本格的野球チーム「新橋アスレチック倶楽部」を設立し、1882年駒場農学校と日本初の対抗戦を行った。なお記録上で日本ではじめて国際試合を行ったのは青井鉞男が投手時代の旧制一高ベースボール部で、1896年(明治29年)5月23日、横浜外人居留地運動場で横浜外人クラブと対戦し29対4で大勝した。また、記録上で日本ではじめて米国人チームと試合を行った(日米野球)のも同部で、同年6月5日、雪辱戦として横浜外人クラブから試合を申し込まれ、横浜外人居留地運動場で当時の米国東洋艦隊の選りすぐりによるオール米国人チームと対戦し32対9で連勝した。

急速な人気の高まりから、野球に対して賛否両論が巻き起こることもあった。1911年に東京朝日新聞が「野球と其害毒」と題した記事を連載し、野球に批判的な著名人の談話などを紹介したが、これに対して読売新聞などが野球擁護の論陣を張り、次第に野球に対するネガティブ・キャンペーンは沈静化していった。

大学野球の盛り上がりは高校(旧制中学)にも広がり、1915年8月に大阪の豊中球場で第1回全国中等学校優勝野球大会が開催され京都二中が優勝。第3回大会からは兵庫の鳴尾球場で開かれたが、観客増により手狭になったため1924年からは阪神電車甲子園大運動場で行われることになった。また夏の大会の盛況をうけ、同年春からは名古屋市の山本球場で全国選抜中等学校野球選手権大会が開催され、翌年からは甲子園球場で行われた。1927年には企業チームによる都市対抗野球大会が明治神宮野球場で開かれた。


\subsection{プロ野球創設}
1920年、早稲田大学野球部OBらによって日本初のプロ野球チーム日本運動協会(芝浦協会)が、1921年には天勝野球団が創設されたが両球団とも後に解散。1934年、読売新聞社の正力松太郎によって大日本東京野球倶楽部が創設され、1936年には日本初のプロ野球リーグ日本職業野球連盟が設立された。

\newpage
\section{醍醐味}
観戦する立場から述べる。野球を観戦するときはたいてい地元のチームを応援して見ている。
野球、もちろん多くの複数人数でプレーするスポーツに共通することであるが、個人の技術力とメンバーの息の合ったプレーがチームの強さとなる。バッターが売ったボールをキャッチャーが捕る。捕ったボールを塁に走者がたどり着く前に近くのチームメンバーに投げる。投げられた側はそれを捕り塁を踏む。この流れの間、走者が塁を踏んでしまうのではないかとプレイヤーもそのチームを応援する観戦者も非常に緊張する。だが、審判のアウトの判定を見て、その緊張は一気に解ける。こういった気持ちの変化分が人間の感じる野球の楽しみなのだろう。
この一連のプレーには個人のキャッチする技術、正確なコントロール、そして投げる側と受け取る側の息の合ったプレーが必要である。

2014年の夏の高校の軟式野球では延長戦が50回に上ることがあった。両者とも1人のピッチャーがこの連戦を投げ続けた。両チームの勝ちへの執念がこのようなことを生んだのだろう。


\section{より盛んになるためには}
野球はテレビで見ている人が多いと思うが、実際に球場に足を運んでもらいその迫力や熱気を感じてもらうことも重要である。それにより、より野球の面白さを理解してもらえるだろう。


\end{document}
